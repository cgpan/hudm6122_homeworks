% Options for packages loaded elsewhere
\PassOptionsToPackage{unicode}{hyperref}
\PassOptionsToPackage{hyphens}{url}
%
\documentclass[
]{article}
\usepackage{amsmath,amssymb}
\usepackage{lmodern}
\usepackage{iftex}
\ifPDFTeX
  \usepackage[T1]{fontenc}
  \usepackage[utf8]{inputenc}
  \usepackage{textcomp} % provide euro and other symbols
\else % if luatex or xetex
  \usepackage{unicode-math}
  \defaultfontfeatures{Scale=MatchLowercase}
  \defaultfontfeatures[\rmfamily]{Ligatures=TeX,Scale=1}
\fi
% Use upquote if available, for straight quotes in verbatim environments
\IfFileExists{upquote.sty}{\usepackage{upquote}}{}
\IfFileExists{microtype.sty}{% use microtype if available
  \usepackage[]{microtype}
  \UseMicrotypeSet[protrusion]{basicmath} % disable protrusion for tt fonts
}{}
\makeatletter
\@ifundefined{KOMAClassName}{% if non-KOMA class
  \IfFileExists{parskip.sty}{%
    \usepackage{parskip}
  }{% else
    \setlength{\parindent}{0pt}
    \setlength{\parskip}{6pt plus 2pt minus 1pt}}
}{% if KOMA class
  \KOMAoptions{parskip=half}}
\makeatother
\usepackage{xcolor}
\usepackage[margin=1in]{geometry}
\usepackage{graphicx}
\makeatletter
\def\maxwidth{\ifdim\Gin@nat@width>\linewidth\linewidth\else\Gin@nat@width\fi}
\def\maxheight{\ifdim\Gin@nat@height>\textheight\textheight\else\Gin@nat@height\fi}
\makeatother
% Scale images if necessary, so that they will not overflow the page
% margins by default, and it is still possible to overwrite the defaults
% using explicit options in \includegraphics[width, height, ...]{}
\setkeys{Gin}{width=\maxwidth,height=\maxheight,keepaspectratio}
% Set default figure placement to htbp
\makeatletter
\def\fps@figure{htbp}
\makeatother
\setlength{\emergencystretch}{3em} % prevent overfull lines
\providecommand{\tightlist}{%
  \setlength{\itemsep}{0pt}\setlength{\parskip}{0pt}}
\setcounter{secnumdepth}{5}
\ifLuaTeX
  \usepackage{selnolig}  % disable illegal ligatures
\fi
\IfFileExists{bookmark.sty}{\usepackage{bookmark}}{\usepackage{hyperref}}
\IfFileExists{xurl.sty}{\usepackage{xurl}}{} % add URL line breaks if available
\urlstyle{same} % disable monospaced font for URLs
\hypersetup{
  pdftitle={HUDM6122 Homework\_05},
  pdfauthor={Chenguang Pan},
  hidelinks,
  pdfcreator={LaTeX via pandoc}}

\title{HUDM6122 Homework\_05}
\author{Chenguang Pan}
\date{2023-03-20}

\begin{document}
\maketitle

\hypertarget{github-address}{%
\subsection{Github Address}\label{github-address}}

All my latest homework can be found on Github:
\url{https://github.com/cgpan/hudm6122_homeworks} . Thanks for checking
if interested.

\hypertarget{ex-5.1}{%
\subsection{Ex 5.1}\label{ex-5.1}}

\emph{Show how the result rises from the assumptions of uncorrelated
factors, independence of the specific variates, and independence of
common factors and specific variances. What form does take if the
factors are allowed to be correlated?}

\textbf{MY SOLUTION:}\\
Based on the assumption of Exploratory Factor Analysis(EFA), a set of
observed variables \textbf{x} assumed to be linked to a set of latent
variables \textbf{f}. Therefore, we can have a regression model in
matrix form
\[\boldsymbol x=\boldsymbol\Lambda \boldsymbol f + \boldsymbol u\],
where \(\boldsymbol \Lambda\) is a \(q \times k\) matrix of factor
loadings (a.k.a., the coefficients of the regression model), and the
\emph{u} is the vector of unexplained error of each observed variables.

Let's take the variance of the formula above
\[V(\boldsymbol x) = V(\boldsymbol\Lambda \boldsymbol f + \boldsymbol u)\].
Based on the operation rule of variance, like
\[V(a+b)= V(a) + V(b) + 2Cov(ab)\], we combined the two formulas above,
then
\[V(\boldsymbol x) = V(\boldsymbol\Lambda \boldsymbol f + \boldsymbol u) = V(\boldsymbol\Lambda \boldsymbol f) + V(\boldsymbol u) + 2 Cov(\boldsymbol\Lambda \boldsymbol f \boldsymbol u)\].
Since the we assumed that the error terms are uncorrelated with the
factors, therefore the
\(Cov(\boldsymbol\Lambda \boldsymbol f \boldsymbol u)=0\). Then, we can
continue to drive the variance formula as
\[V(\boldsymbol x) = V(\boldsymbol\Lambda \boldsymbol f) + V(\boldsymbol u) = \boldsymbol\Lambda V(\boldsymbol f) \boldsymbol\Lambda^T + \Psi\].
In addition, we assumed that the factors are uncorrelated with each
other. The \(V(\boldsymbol f)\) is actually an identity matrix.
Therefore, the formula can be written as
\[V(\boldsymbol x) = \boldsymbol\Lambda V(\boldsymbol f) \boldsymbol\Lambda^T + \Psi = \boldsymbol\Lambda \boldsymbol\Lambda^T + \Psi\].
Finally, the formula can be written as
\[\boldsymbol \Sigma = \boldsymbol\Lambda \boldsymbol\Lambda^T + \Psi\].

If we allow the factors to be correlated with each other, then the
\(V(\boldsymbol f)\) is not an identity matrix. Let's use the greek
letter \(\Phi\) to represent the variance matrix of loadings \textbf{f}.
Thus, the formula should be
\[\boldsymbol \Sigma = \boldsymbol\Lambda \boldsymbol \Phi \boldsymbol\Lambda^T + \Psi\].

\end{document}
