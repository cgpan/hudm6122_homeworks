% Options for packages loaded elsewhere
\PassOptionsToPackage{unicode}{hyperref}
\PassOptionsToPackage{hyphens}{url}
%
\documentclass[
]{article}
\usepackage{amsmath,amssymb}
\usepackage{lmodern}
\usepackage{iftex}
\ifPDFTeX
  \usepackage[T1]{fontenc}
  \usepackage[utf8]{inputenc}
  \usepackage{textcomp} % provide euro and other symbols
\else % if luatex or xetex
  \usepackage{unicode-math}
  \defaultfontfeatures{Scale=MatchLowercase}
  \defaultfontfeatures[\rmfamily]{Ligatures=TeX,Scale=1}
\fi
% Use upquote if available, for straight quotes in verbatim environments
\IfFileExists{upquote.sty}{\usepackage{upquote}}{}
\IfFileExists{microtype.sty}{% use microtype if available
  \usepackage[]{microtype}
  \UseMicrotypeSet[protrusion]{basicmath} % disable protrusion for tt fonts
}{}
\makeatletter
\@ifundefined{KOMAClassName}{% if non-KOMA class
  \IfFileExists{parskip.sty}{%
    \usepackage{parskip}
  }{% else
    \setlength{\parindent}{0pt}
    \setlength{\parskip}{6pt plus 2pt minus 1pt}}
}{% if KOMA class
  \KOMAoptions{parskip=half}}
\makeatother
\usepackage{xcolor}
\usepackage[margin=1in]{geometry}
\usepackage{color}
\usepackage{fancyvrb}
\newcommand{\VerbBar}{|}
\newcommand{\VERB}{\Verb[commandchars=\\\{\}]}
\DefineVerbatimEnvironment{Highlighting}{Verbatim}{commandchars=\\\{\}}
% Add ',fontsize=\small' for more characters per line
\usepackage{framed}
\definecolor{shadecolor}{RGB}{248,248,248}
\newenvironment{Shaded}{\begin{snugshade}}{\end{snugshade}}
\newcommand{\AlertTok}[1]{\textcolor[rgb]{0.94,0.16,0.16}{#1}}
\newcommand{\AnnotationTok}[1]{\textcolor[rgb]{0.56,0.35,0.01}{\textbf{\textit{#1}}}}
\newcommand{\AttributeTok}[1]{\textcolor[rgb]{0.77,0.63,0.00}{#1}}
\newcommand{\BaseNTok}[1]{\textcolor[rgb]{0.00,0.00,0.81}{#1}}
\newcommand{\BuiltInTok}[1]{#1}
\newcommand{\CharTok}[1]{\textcolor[rgb]{0.31,0.60,0.02}{#1}}
\newcommand{\CommentTok}[1]{\textcolor[rgb]{0.56,0.35,0.01}{\textit{#1}}}
\newcommand{\CommentVarTok}[1]{\textcolor[rgb]{0.56,0.35,0.01}{\textbf{\textit{#1}}}}
\newcommand{\ConstantTok}[1]{\textcolor[rgb]{0.00,0.00,0.00}{#1}}
\newcommand{\ControlFlowTok}[1]{\textcolor[rgb]{0.13,0.29,0.53}{\textbf{#1}}}
\newcommand{\DataTypeTok}[1]{\textcolor[rgb]{0.13,0.29,0.53}{#1}}
\newcommand{\DecValTok}[1]{\textcolor[rgb]{0.00,0.00,0.81}{#1}}
\newcommand{\DocumentationTok}[1]{\textcolor[rgb]{0.56,0.35,0.01}{\textbf{\textit{#1}}}}
\newcommand{\ErrorTok}[1]{\textcolor[rgb]{0.64,0.00,0.00}{\textbf{#1}}}
\newcommand{\ExtensionTok}[1]{#1}
\newcommand{\FloatTok}[1]{\textcolor[rgb]{0.00,0.00,0.81}{#1}}
\newcommand{\FunctionTok}[1]{\textcolor[rgb]{0.00,0.00,0.00}{#1}}
\newcommand{\ImportTok}[1]{#1}
\newcommand{\InformationTok}[1]{\textcolor[rgb]{0.56,0.35,0.01}{\textbf{\textit{#1}}}}
\newcommand{\KeywordTok}[1]{\textcolor[rgb]{0.13,0.29,0.53}{\textbf{#1}}}
\newcommand{\NormalTok}[1]{#1}
\newcommand{\OperatorTok}[1]{\textcolor[rgb]{0.81,0.36,0.00}{\textbf{#1}}}
\newcommand{\OtherTok}[1]{\textcolor[rgb]{0.56,0.35,0.01}{#1}}
\newcommand{\PreprocessorTok}[1]{\textcolor[rgb]{0.56,0.35,0.01}{\textit{#1}}}
\newcommand{\RegionMarkerTok}[1]{#1}
\newcommand{\SpecialCharTok}[1]{\textcolor[rgb]{0.00,0.00,0.00}{#1}}
\newcommand{\SpecialStringTok}[1]{\textcolor[rgb]{0.31,0.60,0.02}{#1}}
\newcommand{\StringTok}[1]{\textcolor[rgb]{0.31,0.60,0.02}{#1}}
\newcommand{\VariableTok}[1]{\textcolor[rgb]{0.00,0.00,0.00}{#1}}
\newcommand{\VerbatimStringTok}[1]{\textcolor[rgb]{0.31,0.60,0.02}{#1}}
\newcommand{\WarningTok}[1]{\textcolor[rgb]{0.56,0.35,0.01}{\textbf{\textit{#1}}}}
\usepackage{graphicx}
\makeatletter
\def\maxwidth{\ifdim\Gin@nat@width>\linewidth\linewidth\else\Gin@nat@width\fi}
\def\maxheight{\ifdim\Gin@nat@height>\textheight\textheight\else\Gin@nat@height\fi}
\makeatother
% Scale images if necessary, so that they will not overflow the page
% margins by default, and it is still possible to overwrite the defaults
% using explicit options in \includegraphics[width, height, ...]{}
\setkeys{Gin}{width=\maxwidth,height=\maxheight,keepaspectratio}
% Set default figure placement to htbp
\makeatletter
\def\fps@figure{htbp}
\makeatother
\setlength{\emergencystretch}{3em} % prevent overfull lines
\providecommand{\tightlist}{%
  \setlength{\itemsep}{0pt}\setlength{\parskip}{0pt}}
\setcounter{secnumdepth}{5}
\ifLuaTeX
  \usepackage{selnolig}  % disable illegal ligatures
\fi
\IfFileExists{bookmark.sty}{\usepackage{bookmark}}{\usepackage{hyperref}}
\IfFileExists{xurl.sty}{\usepackage{xurl}}{} % add URL line breaks if available
\urlstyle{same} % disable monospaced font for URLs
\hypersetup{
  pdftitle={HUDM6122 Homework\_05},
  pdfauthor={Chenguang Pan},
  hidelinks,
  pdfcreator={LaTeX via pandoc}}

\title{HUDM6122 Homework\_05}
\author{Chenguang Pan}
\date{2023-03-20}

\begin{document}
\maketitle

\hypertarget{github-address}{%
\subsection{Github Address}\label{github-address}}

All my latest homework can be found on Github:
\url{https://github.com/cgpan/hudm6122_homeworks} . Thanks for checking
if interested.

\hypertarget{ex-5.1}{%
\subsection{Ex 5.1}\label{ex-5.1}}

\emph{Show how the result rises from the assumptions of uncorrelated
factors, independence of the specific variates, and independence of
common factors and specific variances. What form does take if the
factors are allowed to be correlated?}

\textbf{MY SOLUTION:}\\
Based on the assumption of Exploratory Factor Analysis(EFA), a set of
observed variables \textbf{x} assumed to be linked to a set of latent
variables \textbf{f}. Therefore, we can have a regression model in
matrix form
\[\boldsymbol x=\boldsymbol\Lambda \boldsymbol f + \boldsymbol u\],
where \(\boldsymbol \Lambda\) is a \(q \times k\) matrix of factor
loadings (a.k.a., the coefficients of the regression model), and the
\emph{u} is the vector of unexplained error of each observed variables.

Let's take the variance of the formula above
\[V(\boldsymbol x) = V(\boldsymbol\Lambda \boldsymbol f + \boldsymbol u)\].
Based on the operation rule of variance, like
\[V(a+b)= V(a) + V(b) + 2Cov(ab)\], we combined the two formulas above,
then
\[V(\boldsymbol x) = V(\boldsymbol\Lambda \boldsymbol f + \boldsymbol u) = V(\boldsymbol\Lambda \boldsymbol f) + V(\boldsymbol u) + 2 Cov(\boldsymbol\Lambda \boldsymbol f \boldsymbol u)\].
Since the we assumed that the error terms are uncorrelated with the
factors, therefore the
\(Cov(\boldsymbol\Lambda \boldsymbol f \boldsymbol u)=0\). Then, we can
continue to drive the variance formula as
\[V(\boldsymbol x) = V(\boldsymbol\Lambda \boldsymbol f) + V(\boldsymbol u) = \boldsymbol\Lambda V(\boldsymbol f) \boldsymbol\Lambda^T + \Psi\].
In addition, we assumed that the factors are uncorrelated with each
other. The \(V(\boldsymbol f)\) is actually an identity matrix.
Therefore, the formula can be written as
\[V(\boldsymbol x) = \boldsymbol\Lambda V(\boldsymbol f) \boldsymbol\Lambda^T + \Psi = \boldsymbol\Lambda \boldsymbol\Lambda^T + \Psi\].
Finally, the formula can be written as
\[\boldsymbol \Sigma = \boldsymbol\Lambda \boldsymbol\Lambda^T + \Psi\].

If we allow the factors to be correlated with each other, then the
\(V(\boldsymbol f)\) is not an identity matrix. Let's use the greek
letter \(\Phi\) to represent the variance matrix of loadings \textbf{f}.
Thus, the formula should be
\[\boldsymbol \Sigma = \boldsymbol\Lambda \boldsymbol \Phi \boldsymbol\Lambda^T + \Psi\].

\hypertarget{ex-5.2}{%
\subsection{Ex 5.2}\label{ex-5.2}}

\emph{Show that the communalities in a factor analysis model are
unaffected by the transformation \ldots{}}

\textbf{MY SOLUTION:}\\
This question mentioned that we need to use the transformed factor
loadings \(\boldsymbol \Lambda ^* = \boldsymbol \Lambda \boldsymbol M\).
Let's assume that \(\boldsymbol M\) is an \(k \times k\) orthogonal
matrix. We can re-write the the basic regression equation linking the
observed and the factors as:
\[\boldsymbol x=(\boldsymbol\Lambda \boldsymbol M)( \boldsymbol M^T \boldsymbol f) + \boldsymbol u\].\\
Using the rule of variance, we can have
\[\boldsymbol \Sigma = (\boldsymbol\Lambda \boldsymbol M)(\boldsymbol\Lambda \boldsymbol M)^T + \Psi\].
Since the \(\boldsymbol M\) is a orthogonal matrix and
\(\boldsymbol M \boldsymbol M^T = \boldsymbol I\). Therefore, the
variance equation can be written as
\[\boldsymbol \Sigma = \boldsymbol\Lambda \boldsymbol\Lambda^T + \Psi\].
That is, the transformed factor loadings
\(\boldsymbol \Lambda ^* = \boldsymbol \Lambda \boldsymbol M\) will not
influence the communalities (i.e.,
\(\boldsymbol\Lambda \boldsymbol\Lambda^T\)) in the a factor analysis
model.

\hypertarget{ex-5.3}{%
\subsection{Ex 5.3}\label{ex-5.3}}

\emph{Give a formula for the proportion of variance explained by the jth
factor estimated by the principal factor approach.}

\textbf{MY SOLUTION:}\\
The proportion of variance explained by the jth factor represents the
proportion of the total variance in the observed variables that is
accounted for by that factor alone. Therefore, the formula could be
\[Proportion_j = \frac {\sum_{i=1}^{q} \lambda_{ij}^2}{\boldsymbol \Lambda \boldsymbol \Lambda ^T}\].

\hypertarget{ex-5.4}{%
\subsection{Ex 5.4}\label{ex-5.4}}

\emph{Apply the factor analysis model separately to the life
expectancies of men and women and compare the results.}

\textbf{MY SOLUTION:}

For this question, I present two methods to run factor analysis. The
first one is similar to the method introduced in textbook. The second
one is a more rigorous method including the initial dataset checking,
scree plot analysis, and parallel analysis.

\hypertarget{method-1-using-the-similar-method-introduced-by-textbook}{%
\subsubsection{Method 1: Using the similar method introduced by
textbook}\label{method-1-using-the-similar-method-introduced-by-textbook}}

The textbook does not provide the original dataset. Based on the code in
the \texttt{MVA}, I create the dataset via a separated r file named
``HW05 Test''. This file created the \texttt{life.rdata} and
\texttt{life.csv} dataset in the same file folder.

\begin{Shaded}
\begin{Highlighting}[]
\SpecialCharTok{\textgreater{}} \FunctionTok{load}\NormalTok{(}\StringTok{"life.rdata"}\NormalTok{)}
\SpecialCharTok{\textgreater{}} \FunctionTok{head}\NormalTok{(life)}
\NormalTok{           m0 m25 m50 m75 w0 w25 w50 w75}
\NormalTok{Algeria    }\DecValTok{63}  \DecValTok{51}  \DecValTok{30}  \DecValTok{13} \DecValTok{67}  \DecValTok{54}  \DecValTok{34}  \DecValTok{15}
\NormalTok{Cameroon   }\DecValTok{34}  \DecValTok{29}  \DecValTok{13}   \DecValTok{5} \DecValTok{38}  \DecValTok{32}  \DecValTok{17}   \DecValTok{6}
\NormalTok{Madagascar }\DecValTok{38}  \DecValTok{30}  \DecValTok{17}   \DecValTok{7} \DecValTok{38}  \DecValTok{34}  \DecValTok{20}   \DecValTok{7}
\NormalTok{Mauritius  }\DecValTok{59}  \DecValTok{42}  \DecValTok{20}   \DecValTok{6} \DecValTok{64}  \DecValTok{46}  \DecValTok{25}   \DecValTok{8}
\NormalTok{Reunion    }\DecValTok{56}  \DecValTok{38}  \DecValTok{18}   \DecValTok{7} \DecValTok{62}  \DecValTok{46}  \DecValTok{25}  \DecValTok{10}
\NormalTok{Seychelles }\DecValTok{62}  \DecValTok{44}  \DecValTok{24}   \DecValTok{7} \DecValTok{69}  \DecValTok{50}  \DecValTok{28}  \DecValTok{14}
\SpecialCharTok{\textgreater{}} 
\ErrorTok{\textgreater{}} \CommentTok{\# subset the male and female dataset}
\ErrorTok{\textgreater{}}\NormalTok{ life\_male }\OtherTok{\textless{}{-}}\NormalTok{ life[,}\DecValTok{1}\SpecialCharTok{:}\DecValTok{4}\NormalTok{]}
\SpecialCharTok{\textgreater{}}\NormalTok{ life\_female }\OtherTok{\textless{}{-}}\NormalTok{ life[,}\DecValTok{5}\SpecialCharTok{:}\DecValTok{8}\NormalTok{]}
\SpecialCharTok{\textgreater{}} 
\ErrorTok{\textgreater{}} \CommentTok{\# test the number of factors needed for the male and female dataset separately}
\ErrorTok{\textgreater{}} \FunctionTok{sapply}\NormalTok{(}\DecValTok{1}\NormalTok{, }\ControlFlowTok{function}\NormalTok{(f)}
\SpecialCharTok{+}   \FunctionTok{factanal}\NormalTok{(life\_male, }\AttributeTok{factors=}\NormalTok{f, }\AttributeTok{method=}\StringTok{"mle"}\NormalTok{)}\SpecialCharTok{$}\NormalTok{PVAL)}
\NormalTok{   objective }
\FloatTok{0.0007284301} 
\SpecialCharTok{\textgreater{}} \FunctionTok{sapply}\NormalTok{(}\DecValTok{1}\NormalTok{, }\ControlFlowTok{function}\NormalTok{(f)}
\SpecialCharTok{+}   \FunctionTok{factanal}\NormalTok{(life\_female, }\AttributeTok{factors=}\NormalTok{f, }\AttributeTok{method=}\StringTok{"mle"}\NormalTok{)}\SpecialCharTok{$}\NormalTok{PVAL)}
\NormalTok{   objective }
\FloatTok{4.738464e{-}12} 
\end{Highlighting}
\end{Shaded}

When test the number of the factors from 1 to larger number, there is
always a warning that
\texttt{N\ factors\ are\ too\ many\ for\ N\ variables}. More details can
be found on Page 143 of the textbook or here
\url{https://stats.stackexchange.com/questions/593452/efa-n-factors-are-too-many-for-n-variables}

The results suggest that an one-factor solution might be adequate to
account for the observed covariances in the data.

Next, I run the one-factor solution for both male and female datasets.

\begin{Shaded}
\begin{Highlighting}[]
\SpecialCharTok{\textgreater{}} \FunctionTok{factanal}\NormalTok{(life\_male, }\AttributeTok{factors =} \DecValTok{1}\NormalTok{, }\AttributeTok{method=}\StringTok{"mle"}\NormalTok{)}

\NormalTok{Call}\SpecialCharTok{:}
\FunctionTok{factanal}\NormalTok{(}\AttributeTok{x =}\NormalTok{ life\_male, }\AttributeTok{factors =} \DecValTok{1}\NormalTok{, }\AttributeTok{method =} \StringTok{"mle"}\NormalTok{)}

\NormalTok{Uniquenesses}\SpecialCharTok{:}
\NormalTok{   m0   m25   m50   m75 }
\FloatTok{0.594} \FloatTok{0.552} \FloatTok{0.005} \FloatTok{0.434} 

\NormalTok{Loadings}\SpecialCharTok{:}
\NormalTok{    Factor1}
\NormalTok{m0  }\FloatTok{0.638}  
\NormalTok{m25 }\FloatTok{0.669}  
\NormalTok{m50 }\FloatTok{0.998}  
\NormalTok{m75 }\FloatTok{0.752}  

\NormalTok{               Factor1}
\NormalTok{SS loadings      }\FloatTok{2.415}
\NormalTok{Proportion Var   }\FloatTok{0.604}

\NormalTok{Test of the hypothesis that }\DecValTok{1}\NormalTok{ factor is sufficient.}
\NormalTok{The chi square statistic is }\FloatTok{14.45}\NormalTok{ on }\DecValTok{2}\NormalTok{ degrees of freedom.}
\NormalTok{The p}\SpecialCharTok{{-}}\NormalTok{value is }\FloatTok{0.000728} 
\SpecialCharTok{\textgreater{}} \FunctionTok{factanal}\NormalTok{(life\_female, }\AttributeTok{factors =} \DecValTok{1}\NormalTok{, }\AttributeTok{method=}\StringTok{"mle"}\NormalTok{)}

\NormalTok{Call}\SpecialCharTok{:}
\FunctionTok{factanal}\NormalTok{(}\AttributeTok{x =}\NormalTok{ life\_female, }\AttributeTok{factors =} \DecValTok{1}\NormalTok{, }\AttributeTok{method =} \StringTok{"mle"}\NormalTok{)}

\NormalTok{Uniquenesses}\SpecialCharTok{:}
\NormalTok{   w0   w25   w50   w75 }
\FloatTok{0.220} \FloatTok{0.005} \FloatTok{0.115} \FloatTok{0.526} 

\NormalTok{Loadings}\SpecialCharTok{:}
\NormalTok{    Factor1}
\NormalTok{w0  }\FloatTok{0.883}  
\NormalTok{w25 }\FloatTok{0.998}  
\NormalTok{w50 }\FloatTok{0.941}  
\NormalTok{w75 }\FloatTok{0.689}  

\NormalTok{               Factor1}
\NormalTok{SS loadings      }\FloatTok{3.134}
\NormalTok{Proportion Var   }\FloatTok{0.784}

\NormalTok{Test of the hypothesis that }\DecValTok{1}\NormalTok{ factor is sufficient.}
\NormalTok{The chi square statistic is }\FloatTok{52.15}\NormalTok{ on }\DecValTok{2}\NormalTok{ degrees of freedom.}
\NormalTok{The p}\SpecialCharTok{{-}}\NormalTok{value is }\FloatTok{4.74e{-}12} 
\end{Highlighting}
\end{Shaded}

The result shows that for the one-factor solution in the male dataset,
it captures the most variance of age 50 or older. But comparing to the
EFA on complete dataset, this factor does not have a very clear
indication for what properties it covers. The factor analysis on female
dataset has the similar situation.

Is one-factor really appropriate here? Since the P-value is still
significant. I plan to try another method to search for the right number
of factors.

\hypertarget{method-2-solve-this-question-in-another-way}{%
\subsubsection{Method 2: Solve this question in another
way}\label{method-2-solve-this-question-in-another-way}}

Before running EFA, I run several tests to ensure that this dataset is
good for factor analysis.

\begin{Shaded}
\begin{Highlighting}[]
\SpecialCharTok{\textgreater{}} \CommentTok{\# install.packages("psych")}
\ErrorTok{\textgreater{}} \FunctionTok{library}\NormalTok{(psych)}
\SpecialCharTok{\textgreater{}} \CommentTok{\# get the correlation matrix}
\ErrorTok{\textgreater{}}\NormalTok{ life\_male\_cor }\OtherTok{\textless{}{-}} \FunctionTok{cor}\NormalTok{(life\_male)}
\SpecialCharTok{\textgreater{}}\NormalTok{ life\_female\_cor }\OtherTok{\textless{}{-}} \FunctionTok{cor}\NormalTok{(life\_female)}
\SpecialCharTok{\textgreater{}} 
\ErrorTok{\textgreater{}} \CommentTok{\# The Kaiser{-}Meyer{-}Olkin (KMO) used to measure sampling adequacy }
\ErrorTok{\textgreater{}} \CommentTok{\# is a better measure of factorability.}
\ErrorTok{\textgreater{}} \FunctionTok{KMO}\NormalTok{(life\_male\_cor)}
\NormalTok{Kaiser}\SpecialCharTok{{-}}\NormalTok{Meyer}\SpecialCharTok{{-}}\NormalTok{Olkin factor adequacy}
\NormalTok{Call}\SpecialCharTok{:} \FunctionTok{KMO}\NormalTok{(}\AttributeTok{r =}\NormalTok{ life\_male\_cor)}
\NormalTok{Overall MSA }\OtherTok{=}  \FloatTok{0.66}
\NormalTok{MSA }\ControlFlowTok{for}\NormalTok{ each item }\OtherTok{=} 
\NormalTok{  m0  m25  m50  m75 }
\FloatTok{0.66} \FloatTok{0.77} \FloatTok{0.64} \FloatTok{0.55} 
\SpecialCharTok{\textgreater{}} \FunctionTok{KMO}\NormalTok{(life\_female\_cor)}
\NormalTok{Kaiser}\SpecialCharTok{{-}}\NormalTok{Meyer}\SpecialCharTok{{-}}\NormalTok{Olkin factor adequacy}
\NormalTok{Call}\SpecialCharTok{:} \FunctionTok{KMO}\NormalTok{(}\AttributeTok{r =}\NormalTok{ life\_female\_cor)}
\NormalTok{Overall MSA }\OtherTok{=}  \FloatTok{0.63}
\NormalTok{MSA }\ControlFlowTok{for}\NormalTok{ each item }\OtherTok{=} 
\NormalTok{  w0  w25  w50  w75 }
\FloatTok{0.54} \FloatTok{0.59} \FloatTok{0.64} \FloatTok{0.82} 
\end{Highlighting}
\end{Shaded}

According to Kaiser's (1974) guidelines, a suggested cutoff for
determining the factorability of the sample data is KMO \textgreater=
60. The total KMOs are 0.66 and 0.63, indicating that, based on this
test, we can probably conduct a factor analysis.

Next, Bartlett's Test of Sphericity compares an observed correlation
matrix to the identity matrix. Essentially it checks to see if there is
a certain redundancy between the variables that we can summarize with a
few number of factors. The null hypothesis of the test is that the
variables are orthogonal, i.e.~not correlated.

\begin{Shaded}
\begin{Highlighting}[]
\SpecialCharTok{\textgreater{}} \CommentTok{\# run Bartlett\textquotesingle{}s Test of Sphericity}
\ErrorTok{\textgreater{}} \FunctionTok{cortest.bartlett}\NormalTok{(life\_male\_cor)}\SpecialCharTok{$}\NormalTok{p.value}
\NormalTok{[}\DecValTok{1}\NormalTok{] }\FloatTok{8.437942e{-}49}
\SpecialCharTok{\textgreater{}} \FunctionTok{cortest.bartlett}\NormalTok{(life\_female\_cor)}\SpecialCharTok{$}\NormalTok{p.value}
\NormalTok{[}\DecValTok{1}\NormalTok{] }\FloatTok{9.75451e{-}127}
\end{Highlighting}
\end{Shaded}

Small p values (\textless{} 0.05) of the significance level indicate
that a factor analysis may be useful with our data.

\begin{Shaded}
\begin{Highlighting}[]
\SpecialCharTok{\textgreater{}} \CommentTok{\# get the determinants for both correlation matrix}
\ErrorTok{\textgreater{}} \FunctionTok{det}\NormalTok{(life\_male\_cor)}
\NormalTok{[}\DecValTok{1}\NormalTok{] }\FloatTok{0.08459114}
\SpecialCharTok{\textgreater{}} \FunctionTok{det}\NormalTok{(life\_female\_cor)}
\NormalTok{[}\DecValTok{1}\NormalTok{] }\FloatTok{0.002000542}
\end{Highlighting}
\end{Shaded}

Finally, we have positive determinants, which means the factor analysis
will probably run.

Here, I begin to run EFA by using \texttt{fa()} function and make a
scree plot to determine the number of factors.

\begin{Shaded}
\begin{Highlighting}[]
\SpecialCharTok{\textgreater{}} \FunctionTok{library}\NormalTok{(ggplot2)}
\SpecialCharTok{\textgreater{}} \CommentTok{\# run factor analysis using fa() function}
\ErrorTok{\textgreater{}}\NormalTok{ male\_fa }\OtherTok{\textless{}{-}} \FunctionTok{fa}\NormalTok{(life\_male, }
\SpecialCharTok{+}               \AttributeTok{nfactors =} \FunctionTok{ncol}\NormalTok{(life\_male\_cor), }
\SpecialCharTok{+}               \AttributeTok{rotate =} \StringTok{"varimax"}\NormalTok{)}
\SpecialCharTok{\textgreater{}}\NormalTok{ efa\_model }\OtherTok{\textless{}{-}} \FunctionTok{fa}\NormalTok{(life\_male, }\AttributeTok{nfactors =} \DecValTok{2}\NormalTok{, }\AttributeTok{rotate =} \StringTok{"varimax"}\NormalTok{)}
\SpecialCharTok{\textgreater{}} \CommentTok{\# to get the number of factors}
\ErrorTok{\textgreater{}}\NormalTok{ n\_factors }\OtherTok{\textless{}{-}} \FunctionTok{length}\NormalTok{(male\_fa}\SpecialCharTok{$}\NormalTok{e.values)}
\SpecialCharTok{\textgreater{}} 
\ErrorTok{\textgreater{}} \CommentTok{\# to store the data}
\ErrorTok{\textgreater{}}\NormalTok{ scree }\OtherTok{\textless{}{-}} \FunctionTok{data.frame}\NormalTok{(}\AttributeTok{Factor\_n =} \FunctionTok{as.factor}\NormalTok{(}\DecValTok{1}\SpecialCharTok{:}\NormalTok{n\_factors), }
\SpecialCharTok{+}                     \AttributeTok{Eigenvalue =}\NormalTok{ male\_fa}\SpecialCharTok{$}\NormalTok{e.values)}
\SpecialCharTok{\textgreater{}} \CommentTok{\# draw scree plot using ggplot2}
\ErrorTok{\textgreater{}} \FunctionTok{ggplot}\NormalTok{(scree, }\FunctionTok{aes}\NormalTok{(}\AttributeTok{x =}\NormalTok{ Factor\_n, }\AttributeTok{y =}\NormalTok{ Eigenvalue, }\AttributeTok{group =} \DecValTok{1}\NormalTok{)) }\SpecialCharTok{+} 
\SpecialCharTok{+}       \FunctionTok{geom\_point}\NormalTok{() }\SpecialCharTok{+} \FunctionTok{geom\_line}\NormalTok{() }\SpecialCharTok{+}
\SpecialCharTok{+}       \FunctionTok{xlab}\NormalTok{(}\StringTok{"Number of factors"}\NormalTok{) }\SpecialCharTok{+}
\SpecialCharTok{+}       \FunctionTok{ylab}\NormalTok{(}\StringTok{"Initial eigenvalue"}\NormalTok{) }\SpecialCharTok{+}
\SpecialCharTok{+}       \FunctionTok{labs}\NormalTok{( }\AttributeTok{title =} \StringTok{"Scree Plot"}\NormalTok{, }
\SpecialCharTok{+}             \AttributeTok{subtitle =} \StringTok{"(Based on the unreduced correlation matrix)"}\NormalTok{)}
\end{Highlighting}
\end{Shaded}

\includegraphics{HUDM6122-Homework_05-Chenguang-Pan_files/figure-latex/unnamed-chunk-6-1.pdf}
From the scree plot, 1 factors maybe appropriate for the male dataset.
Since only one factor's eigenvalue is greater than 1. However, the
second factor is above .7. It may be appropriate too.

\emph{Tips} Why I set eigenvalue = 1 as a cutoff?\\
Here, the eigenvalue is a measure of the amount of variance in the
observed variables that is accounted for by each factor. If the
eigenvalue of a factor is less than 1, it indicates that the factor
explains less variance than one of the original variables and,
therefore, does not contribute significantly to the explanation of the
common variance among the variables.

Parallel analysis is a method for determining the number of components
or factors to retain from pca or factor analysis.I also use perform
parallel analysis to determine the factors. notice the results in the
console will provide the suggestion.

From the returned result under the scree plot, it suggests that 2-factor
solution may be good for the male dataset.

\begin{Shaded}
\begin{Highlighting}[]
\SpecialCharTok{\textgreater{}}\NormalTok{ parallel }\OtherTok{\textless{}{-}} \FunctionTok{fa.parallel}\NormalTok{(life\_male\_cor)}
\end{Highlighting}
\end{Shaded}

\includegraphics{HUDM6122-Homework_05-Chenguang-Pan_files/figure-latex/unnamed-chunk-7-1.pdf}

\begin{verbatim}
Parallel analysis suggests that the number of factors =  2  and the number of components =  1 
\end{verbatim}

Using the same method on the female dataset.

\begin{Shaded}
\begin{Highlighting}[]
\SpecialCharTok{\textgreater{}} \CommentTok{\# run factor analysis using fa() function}
\ErrorTok{\textgreater{}}\NormalTok{ female\_fa }\OtherTok{\textless{}{-}} \FunctionTok{fa}\NormalTok{(life\_female, }
\SpecialCharTok{+}               \AttributeTok{nfactors =} \FunctionTok{ncol}\NormalTok{(life\_female\_cor), }
\SpecialCharTok{+}               \AttributeTok{rotate =} \StringTok{"varimax"}\NormalTok{)}
\SpecialCharTok{\textgreater{}} \CommentTok{\# to get the number of factors}
\ErrorTok{\textgreater{}}\NormalTok{ n\_factors }\OtherTok{\textless{}{-}} \FunctionTok{length}\NormalTok{(female\_fa}\SpecialCharTok{$}\NormalTok{e.values)}
\SpecialCharTok{\textgreater{}} 
\ErrorTok{\textgreater{}} \CommentTok{\# to store the data}
\ErrorTok{\textgreater{}}\NormalTok{ scree }\OtherTok{\textless{}{-}} \FunctionTok{data.frame}\NormalTok{(}\AttributeTok{Factor\_n =} \FunctionTok{as.factor}\NormalTok{(}\DecValTok{1}\SpecialCharTok{:}\NormalTok{n\_factors), }
\SpecialCharTok{+}                     \AttributeTok{Eigenvalue =}\NormalTok{ female\_fa}\SpecialCharTok{$}\NormalTok{e.values)}
\SpecialCharTok{\textgreater{}} \FunctionTok{par}\NormalTok{(}\AttributeTok{mfrow=}\FunctionTok{c}\NormalTok{(}\DecValTok{1}\NormalTok{, }\DecValTok{2}\NormalTok{))}
\SpecialCharTok{\textgreater{}} \CommentTok{\# draw scree plot using ggplot2}
\ErrorTok{\textgreater{}} \FunctionTok{ggplot}\NormalTok{(scree, }\FunctionTok{aes}\NormalTok{(}\AttributeTok{x =}\NormalTok{ Factor\_n, }\AttributeTok{y =}\NormalTok{ Eigenvalue, }\AttributeTok{group =} \DecValTok{1}\NormalTok{)) }\SpecialCharTok{+} 
\SpecialCharTok{+}       \FunctionTok{geom\_point}\NormalTok{() }\SpecialCharTok{+} \FunctionTok{geom\_line}\NormalTok{() }\SpecialCharTok{+}
\SpecialCharTok{+}       \FunctionTok{xlab}\NormalTok{(}\StringTok{"Number of factors"}\NormalTok{) }\SpecialCharTok{+}
\SpecialCharTok{+}       \FunctionTok{ylab}\NormalTok{(}\StringTok{"Initial eigenvalue"}\NormalTok{) }\SpecialCharTok{+}
\SpecialCharTok{+}       \FunctionTok{labs}\NormalTok{( }\AttributeTok{title =} \StringTok{"Scree Plot"}\NormalTok{, }
\SpecialCharTok{+}             \AttributeTok{subtitle =} \StringTok{"(Based on the unreduced correlation matrix)"}\NormalTok{)}
\SpecialCharTok{\textgreater{}} 
\ErrorTok{\textgreater{}}\NormalTok{ parallel }\OtherTok{\textless{}{-}} \FunctionTok{fa.parallel}\NormalTok{(life\_male\_cor)}
\NormalTok{Parallel analysis suggests that the number of factors }\OtherTok{=}  \DecValTok{2}\NormalTok{  and the number of components }\OtherTok{=}  \DecValTok{1} 
\end{Highlighting}
\end{Shaded}

\includegraphics[width=0.5\linewidth,height=0.5\textheight]{HUDM6122-Homework_05-Chenguang-Pan_files/figure-latex/unnamed-chunk-8-1}
\includegraphics[width=0.5\linewidth,height=0.5\textheight]{HUDM6122-Homework_05-Chenguang-Pan_files/figure-latex/unnamed-chunk-8-2}
The results also suggest that 2-factor solution is good for female
dataset.

Finally, I checked the factor loadings of both male and female datasets.

\begin{Shaded}
\begin{Highlighting}[]
\SpecialCharTok{\textgreater{}}\NormalTok{ male\_fa}\SpecialCharTok{$}\NormalTok{loadings}

\NormalTok{Loadings}\SpecialCharTok{:}
\NormalTok{    MR1    MR2    MR3    MR4   }
\NormalTok{m0   }\FloatTok{0.869}  \FloatTok{0.189}  \FloatTok{0.104}       
\NormalTok{m25  }\FloatTok{0.803}  \FloatTok{0.310}              
\NormalTok{m50  }\FloatTok{0.540}  \FloatTok{0.792}  \FloatTok{0.152}       
\NormalTok{m75  }\FloatTok{0.153}  \FloatTok{0.853}              

\NormalTok{                 MR1   MR2   MR3   MR4}
\NormalTok{SS loadings    }\FloatTok{1.716} \FloatTok{1.488} \FloatTok{0.043} \FloatTok{0.000}
\NormalTok{Proportion Var }\FloatTok{0.429} \FloatTok{0.372} \FloatTok{0.011} \FloatTok{0.000}
\NormalTok{Cumulative Var }\FloatTok{0.429} \FloatTok{0.801} \FloatTok{0.812} \FloatTok{0.812}
\SpecialCharTok{\textgreater{}}\NormalTok{ female\_fa}\SpecialCharTok{$}\NormalTok{loadings}

\NormalTok{Loadings}\SpecialCharTok{:}
\NormalTok{    MR1    MR2    MR3    MR4   }
\NormalTok{w0   }\FloatTok{0.958}  \FloatTok{0.210}              
\NormalTok{w25  }\FloatTok{0.793}  \FloatTok{0.605}              
\NormalTok{w50  }\FloatTok{0.560}  \FloatTok{0.818}              
\NormalTok{w75  }\FloatTok{0.187}  \FloatTok{0.887}              

\NormalTok{                 MR1   MR2   MR3   MR4}
\NormalTok{SS loadings    }\FloatTok{1.896} \FloatTok{1.866} \FloatTok{0.009} \FloatTok{0.000}
\NormalTok{Proportion Var }\FloatTok{0.474} \FloatTok{0.466} \FloatTok{0.002} \FloatTok{0.000}
\NormalTok{Cumulative Var }\FloatTok{0.474} \FloatTok{0.940} \FloatTok{0.942} \FloatTok{0.942}
\end{Highlighting}
\end{Shaded}

This two-factor solution looks more reasonable than the one-factor.
Since in both dataset, the first factor captures the 0-25 age's life
expectations and the second factor covers the 50-75 age's life
expectations. We can call the first factor ``Life force under middle
age'' and the second factor ``life force of or above middle age''.

\hypertarget{ex-5.5}{%
\subsection{Ex 5.5}\label{ex-5.5}}

\emph{The correlation matrix given below arises from the scores of 220
boys in six school subjects: (1) French, (2) English, (3) History, (4)
Arithmetic, (5) Algebra, and (6) Geometry. Find the two-factor solution
from a maximum likelihood factor analysis. By plotting the derived
loadings, find an orthogonal rotation that allows easier interpretation
of the results.}

\textbf{MY SOLUTION:}

\begin{Shaded}
\begin{Highlighting}[]
\SpecialCharTok{\textgreater{}} \CommentTok{\# import the data}
\ErrorTok{\textgreater{}} \FunctionTok{library}\NormalTok{(Matrix)}
\SpecialCharTok{\textgreater{}} \CommentTok{\# import the correlation matrix}
\ErrorTok{\textgreater{}}\NormalTok{ corr\_lower }\OtherTok{\textless{}{-}} \FunctionTok{matrix}\NormalTok{(}\FunctionTok{c}\NormalTok{(}\DecValTok{1}\NormalTok{, }\DecValTok{0}\NormalTok{, }\DecValTok{0}\NormalTok{, }\DecValTok{0}\NormalTok{, }\DecValTok{0}\NormalTok{, }\DecValTok{0}\NormalTok{, }
\SpecialCharTok{+}                       \FloatTok{0.44}\NormalTok{, }\DecValTok{1}\NormalTok{, }\DecValTok{0}\NormalTok{,}\DecValTok{0}\NormalTok{,}\DecValTok{0}\NormalTok{,}\DecValTok{0}\NormalTok{,}
\SpecialCharTok{+}                       \FloatTok{0.41}\NormalTok{, }\FloatTok{0.35}\NormalTok{, }\DecValTok{1}\NormalTok{,}\DecValTok{0}\NormalTok{,}\DecValTok{0}\NormalTok{,}\DecValTok{0}\NormalTok{,}
\SpecialCharTok{+}                       \FloatTok{0.29}\NormalTok{, }\FloatTok{0.35}\NormalTok{, }\FloatTok{0.16}\NormalTok{, }\DecValTok{1}\NormalTok{,}\DecValTok{0}\NormalTok{,}\DecValTok{0}\NormalTok{,}
\SpecialCharTok{+}                       \FloatTok{0.33}\NormalTok{, }\FloatTok{0.32}\NormalTok{, }\FloatTok{0.19}\NormalTok{, }\FloatTok{0.59}\NormalTok{, }\DecValTok{1}\NormalTok{,}\DecValTok{0}\NormalTok{,}
\SpecialCharTok{+}                       \FloatTok{0.25}\NormalTok{, }\FloatTok{0.33}\NormalTok{, }\FloatTok{0.18}\NormalTok{, }\FloatTok{0.47}\NormalTok{, }\FloatTok{0.46}\NormalTok{, }\DecValTok{1}\NormalTok{),}\DecValTok{6}\NormalTok{,}\DecValTok{6}\NormalTok{, }\AttributeTok{byrow =}\NormalTok{ T)}
\SpecialCharTok{\textgreater{}} \CommentTok{\# generate a complete correlation matrix}
\ErrorTok{\textgreater{}}\NormalTok{ corr\_symmetric }\OtherTok{\textless{}{-}} \FunctionTok{forceSymmetric}\NormalTok{(corr\_lower, }\AttributeTok{uplo=}\StringTok{"L"}\NormalTok{)}
\SpecialCharTok{\textgreater{}} \FunctionTok{class}\NormalTok{(corr\_symmetric)}
\NormalTok{[}\DecValTok{1}\NormalTok{] }\StringTok{"dsyMatrix"}
\FunctionTok{attr}\NormalTok{(,}\StringTok{"package"}\NormalTok{)}
\NormalTok{[}\DecValTok{1}\NormalTok{] }\StringTok{"Matrix"}
\end{Highlighting}
\end{Shaded}

Although the question has told that we can use two-factor solution to
conduct factor analysis, I still checked the right number of factors

\begin{Shaded}
\begin{Highlighting}[]
\SpecialCharTok{\textgreater{}} \CommentTok{\# test for the right number of factors}
\ErrorTok{\textgreater{}} \FunctionTok{sapply}\NormalTok{(}\DecValTok{1}\SpecialCharTok{:}\DecValTok{2}\NormalTok{, }\ControlFlowTok{function}\NormalTok{(f) }
\SpecialCharTok{+}   \FunctionTok{factanal}\NormalTok{(}\AttributeTok{covmat=}\FunctionTok{as.matrix}\NormalTok{(corr\_symmetric), }\AttributeTok{factors =}\NormalTok{ f, }
\SpecialCharTok{+}            \AttributeTok{method =} \StringTok{"mle"}\NormalTok{, }\AttributeTok{n.obs =} \DecValTok{220}\NormalTok{)}\SpecialCharTok{$}\NormalTok{PVAL)}
\NormalTok{   objective    objective }
\FloatTok{5.369962e{-}08} \FloatTok{7.026721e{-}01} 
\end{Highlighting}
\end{Shaded}

The result shows that the two-factor solution is adequate here. The
result from the two-factor varimax solution are obtained from

\begin{Shaded}
\begin{Highlighting}[]
\SpecialCharTok{\textgreater{}}\NormalTok{ fa\_ }\OtherTok{\textless{}{-}} \FunctionTok{factanal}\NormalTok{(}\AttributeTok{covmat =} \FunctionTok{as.matrix}\NormalTok{(corr\_symmetric), }\AttributeTok{factors =} \DecValTok{2}\NormalTok{,}
\SpecialCharTok{+}           \AttributeTok{method=}\StringTok{"mle"}\NormalTok{, }\AttributeTok{n.obs =} \DecValTok{220}\NormalTok{)}
\SpecialCharTok{\textgreater{}}\NormalTok{ fa\_}\SpecialCharTok{$}\NormalTok{loadings}

\NormalTok{Loadings}\SpecialCharTok{:}
\NormalTok{     Factor1 Factor2}
\NormalTok{[}\DecValTok{1}\NormalTok{,] }\FloatTok{0.233}   \FloatTok{0.661}  
\NormalTok{[}\DecValTok{2}\NormalTok{,] }\FloatTok{0.319}   \FloatTok{0.551}  
\NormalTok{[}\DecValTok{3}\NormalTok{,]         }\FloatTok{0.591}  
\NormalTok{[}\DecValTok{4}\NormalTok{,] }\FloatTok{0.770}   \FloatTok{0.172}  
\NormalTok{[}\DecValTok{5}\NormalTok{,] }\FloatTok{0.715}   \FloatTok{0.220}  
\NormalTok{[}\DecValTok{6}\NormalTok{,] }\FloatTok{0.570}   \FloatTok{0.215}  

\NormalTok{               Factor1 Factor2}
\NormalTok{SS loadings      }\FloatTok{1.593}   \FloatTok{1.215}
\NormalTok{Proportion Var   }\FloatTok{0.265}   \FloatTok{0.202}
\NormalTok{Cumulative Var   }\FloatTok{0.265}   \FloatTok{0.468}
\SpecialCharTok{\textgreater{}} 
\ErrorTok{\textgreater{}} \CommentTok{\# plot the derived loadings}
\ErrorTok{\textgreater{}}\NormalTok{ loadings }\OtherTok{\textless{}{-}}\NormalTok{ fa\_}\SpecialCharTok{$}\NormalTok{loadings[,}\DecValTok{1}\SpecialCharTok{:}\DecValTok{2}\NormalTok{]}
\SpecialCharTok{\textgreater{}} \FunctionTok{plot}\NormalTok{(loadings[,}\DecValTok{1}\NormalTok{], loadings[,}\DecValTok{2}\NormalTok{], }
\SpecialCharTok{+}      \AttributeTok{type=}\StringTok{"n"}\NormalTok{,}\AttributeTok{xlab=}\StringTok{"Factor 1"}\NormalTok{,}\AttributeTok{ylab=}\StringTok{"Factor 2"}\NormalTok{)}
\SpecialCharTok{\textgreater{}} \FunctionTok{text}\NormalTok{(loadings[,}\DecValTok{1}\NormalTok{], loadings[,}\DecValTok{2}\NormalTok{],}
\SpecialCharTok{+}      \FunctionTok{abbreviate}\NormalTok{(}\FunctionTok{c}\NormalTok{(}\StringTok{"French"}\NormalTok{,}\StringTok{"English"}\NormalTok{,}\StringTok{"History"}\NormalTok{,}
\SpecialCharTok{+}                   \StringTok{"Arithmetic"}\NormalTok{,}\StringTok{"Algebra"}\NormalTok{,}\StringTok{"Geometry"}\NormalTok{),}\DecValTok{3}\NormalTok{),}\AttributeTok{cex=}\FloatTok{0.7}\NormalTok{)}
\end{Highlighting}
\end{Shaded}

\includegraphics{HUDM6122-Homework_05-Chenguang-Pan_files/figure-latex/unnamed-chunk-12-1.pdf}

\hypertarget{ex-5.6}{%
\subsection{Ex 5.6}\label{ex-5.6}}

\emph{The matrix below shows the correlations between ratings on nine
statements about pain made by 123 people suffering from extreme pain.
Each statement was scored on a scale from 1 to 6, ranging from agreement
to disagreement. The nine pain statements were as follows:}

\textbf{MY SOLUTION:}\\
First, to change the lower triangular matrix into the complete
correlation matrix.

\begin{Shaded}
\begin{Highlighting}[]
\SpecialCharTok{\textgreater{}} \FunctionTok{library}\NormalTok{(Matrix)}
\SpecialCharTok{\textgreater{}} \CommentTok{\# import the correlation matrix}
\ErrorTok{\textgreater{}}\NormalTok{ corr\_lower }\OtherTok{\textless{}{-}} \FunctionTok{matrix}\NormalTok{(}\FunctionTok{c}\NormalTok{(}\DecValTok{1}\NormalTok{, }\DecValTok{0}\NormalTok{, }\DecValTok{0}\NormalTok{, }\DecValTok{0}\NormalTok{, }\DecValTok{0}\NormalTok{, }\DecValTok{0}\NormalTok{, }\DecValTok{0}\NormalTok{, }\DecValTok{0}\NormalTok{, }\DecValTok{0}\NormalTok{,}
\SpecialCharTok{+}                       \SpecialCharTok{{-}}\FloatTok{0.04}\NormalTok{, }\DecValTok{1}\NormalTok{, }\DecValTok{0}\NormalTok{,}\DecValTok{0}\NormalTok{,}\DecValTok{0}\NormalTok{,}\DecValTok{0}\NormalTok{,}\DecValTok{0}\NormalTok{,}\DecValTok{0}\NormalTok{,}\DecValTok{0}\NormalTok{,}
\SpecialCharTok{+}                       \FloatTok{0.61}\NormalTok{, }\SpecialCharTok{{-}}\FloatTok{0.07}\NormalTok{, }\DecValTok{1}\NormalTok{,}\DecValTok{0}\NormalTok{,}\DecValTok{0}\NormalTok{,}\DecValTok{0}\NormalTok{,}\DecValTok{0}\NormalTok{,}\DecValTok{0}\NormalTok{,}\DecValTok{0}\NormalTok{,}
\SpecialCharTok{+}                       \FloatTok{0.45}\NormalTok{, }\SpecialCharTok{{-}}\FloatTok{0.12}\NormalTok{, }\FloatTok{0.59}\NormalTok{, }\DecValTok{1}\NormalTok{,}\DecValTok{0}\NormalTok{,}\DecValTok{0}\NormalTok{,}\DecValTok{0}\NormalTok{,}\DecValTok{0}\NormalTok{,}\DecValTok{0}\NormalTok{,}
\SpecialCharTok{+}                       \FloatTok{0.03}\NormalTok{, }\FloatTok{0.49}\NormalTok{, }\FloatTok{0.03}\NormalTok{, }\SpecialCharTok{{-}}\FloatTok{0.08}\NormalTok{, }\DecValTok{1}\NormalTok{,}\DecValTok{0}\NormalTok{,}\DecValTok{0}\NormalTok{,}\DecValTok{0}\NormalTok{,}\DecValTok{0}\NormalTok{,}
\SpecialCharTok{+}                       \SpecialCharTok{{-}}\FloatTok{0.29}\NormalTok{, }\FloatTok{0.43}\NormalTok{, }\SpecialCharTok{{-}}\FloatTok{0.13}\NormalTok{, }\SpecialCharTok{{-}}\FloatTok{0.21}\NormalTok{, }\FloatTok{0.47}\NormalTok{, }\DecValTok{1}\NormalTok{,}\DecValTok{0}\NormalTok{,}\DecValTok{0}\NormalTok{,}\DecValTok{0}\NormalTok{,}
\SpecialCharTok{+}                       \SpecialCharTok{{-}}\FloatTok{0.30}\NormalTok{, }\FloatTok{0.30}\NormalTok{, }\SpecialCharTok{{-}}\FloatTok{0.24}\NormalTok{, }\SpecialCharTok{{-}}\FloatTok{0.19}\NormalTok{, }\FloatTok{0.41}\NormalTok{, }\FloatTok{0.63}\NormalTok{,}\DecValTok{1}\NormalTok{,}\DecValTok{0}\NormalTok{,}\DecValTok{0}\NormalTok{, }
\SpecialCharTok{+}                       \FloatTok{0.45}\NormalTok{, }\SpecialCharTok{{-}}\FloatTok{0.31}\NormalTok{,}\FloatTok{0.59}\NormalTok{,}\FloatTok{0.63}\NormalTok{,}\SpecialCharTok{{-}}\FloatTok{0.14}\NormalTok{,}\SpecialCharTok{{-}}\FloatTok{0.13}\NormalTok{,}\SpecialCharTok{{-}}\FloatTok{0.26}\NormalTok{,}\DecValTok{1}\NormalTok{,}\DecValTok{0}\NormalTok{,}
\SpecialCharTok{+}                       \FloatTok{0.30}\NormalTok{,}\SpecialCharTok{{-}}\FloatTok{0.17}\NormalTok{,}\SpecialCharTok{{-}}\FloatTok{0.32}\NormalTok{,}\FloatTok{0.37}\NormalTok{,}\SpecialCharTok{{-}}\FloatTok{0.24}\NormalTok{,}\SpecialCharTok{{-}}\FloatTok{0.15}\NormalTok{,}\SpecialCharTok{{-}}\FloatTok{0.29}\NormalTok{,}\FloatTok{0.40}\NormalTok{,}\DecValTok{1}\NormalTok{),}\DecValTok{9}\NormalTok{,}\DecValTok{9}\NormalTok{, }\AttributeTok{byrow =}\NormalTok{ T)}
\SpecialCharTok{\textgreater{}} \CommentTok{\# generate a complete correlation matrix}
\ErrorTok{\textgreater{}}\NormalTok{ corr\_symmetric }\OtherTok{\textless{}{-}} \FunctionTok{forceSymmetric}\NormalTok{(corr\_lower, }\AttributeTok{uplo=}\StringTok{"L"}\NormalTok{)}
\end{Highlighting}
\end{Shaded}

The correlation matrix looks good. Next, I run the PCA first.

\begin{Shaded}
\begin{Highlighting}[]
\SpecialCharTok{\textgreater{}} \CommentTok{\# run the PCA first}
\ErrorTok{\textgreater{}} \CommentTok{\# use prcomp to calculate the principal components}
\ErrorTok{\textgreater{}}\NormalTok{ pca }\OtherTok{\textless{}{-}} \FunctionTok{prcomp}\NormalTok{(corr\_symmetric, }\AttributeTok{scale. =} \ConstantTok{FALSE}\NormalTok{)}
\SpecialCharTok{\textgreater{}} \CommentTok{\# get the PCA results}
\ErrorTok{\textgreater{}} \FunctionTok{summary}\NormalTok{(pca)}
\NormalTok{Importance of components}\SpecialCharTok{:}
\NormalTok{                          PC1    PC2     PC3     PC4    PC5     PC6     PC7}
\NormalTok{Standard deviation     }\FloatTok{1.1594} \FloatTok{0.5014} \FloatTok{0.31822} \FloatTok{0.18698} \FloatTok{0.1734} \FloatTok{0.15725} \FloatTok{0.09446}
\NormalTok{Proportion of Variance }\FloatTok{0.7466} \FloatTok{0.1396} \FloatTok{0.05624} \FloatTok{0.01942} \FloatTok{0.0167} \FloatTok{0.01373} \FloatTok{0.00496}
\NormalTok{Cumulative Proportion  }\FloatTok{0.7466} \FloatTok{0.8863} \FloatTok{0.94250} \FloatTok{0.96192} \FloatTok{0.9786} \FloatTok{0.99235} \FloatTok{0.99730}
\NormalTok{                           PC8       PC9}
\NormalTok{Standard deviation     }\FloatTok{0.06968} \FloatTok{3.631e{-}17}
\NormalTok{Proportion of Variance }\FloatTok{0.00270} \FloatTok{0.000e+00}
\NormalTok{Cumulative Proportion  }\FloatTok{1.00000} \FloatTok{1.000e+00}
\SpecialCharTok{\textgreater{}} \CommentTok{\# draw the scree plot }
\ErrorTok{\textgreater{}} \FunctionTok{plot}\NormalTok{(pca, }\AttributeTok{type =} \StringTok{"l"}\NormalTok{, }
\SpecialCharTok{+}      \AttributeTok{main =} \StringTok{"Scree Plot"}\NormalTok{)}
\end{Highlighting}
\end{Shaded}

\includegraphics{HUDM6122-Homework_05-Chenguang-Pan_files/figure-latex/unnamed-chunk-14-1.pdf}
The scree-plot shows that 3 principle components may be appropriate.
However, based on the results from the PCA analysis, the first two
components can explain 88.63\% variance of the total. Therefore, I
choose the first two to represent the data.

Next, I run maximum likelihood factor analysis.

\begin{Shaded}
\begin{Highlighting}[]
\SpecialCharTok{\textgreater{}} \CommentTok{\# explore the number of factors}
\ErrorTok{\textgreater{}} \FunctionTok{as.matrix}\NormalTok{(corr\_symmetric)}
\SpecialCharTok{\textgreater{}} \FunctionTok{sapply}\NormalTok{(}\DecValTok{1}\SpecialCharTok{:}\DecValTok{6}\NormalTok{, }\ControlFlowTok{function}\NormalTok{(f) }\FunctionTok{factanal}\NormalTok{(}\AttributeTok{covmat=}\FunctionTok{as.matrix}\NormalTok{(corr\_symmetric), }\AttributeTok{factors=}\NormalTok{f, }\AttributeTok{method=}\StringTok{"mle"}\NormalTok{, }\AttributeTok{n.obs =} \DecValTok{123}\NormalTok{)}\SpecialCharTok{$}\NormalTok{PVAL)}
\SpecialCharTok{\textgreater{}} 
\end{Highlighting}
\end{Shaded}


\end{document}
